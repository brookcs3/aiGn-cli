\documentclass[11pt]{article}

% =====================
% Packages
% =====================
\usepackage[margin=1in]{geometry}
\usepackage[T1]{fontenc}
\usepackage[utf8]{inputenc}
\usepackage{lmodern}
\usepackage{microtype}
\usepackage{xcolor}
\usepackage{graphicx}
\usepackage{float}
\usepackage{setspace}
\usepackage{enumitem}
\usepackage{booktabs}
\usepackage{fancyhdr}
\usepackage{hyperref}

% =====================
% Colors & Styling
% =====================
\definecolor{osuorange}{HTML}{D73F09}
\definecolor{ink}{HTML}{222222}
\definecolor{lightgray}{HTML}{F5F5F5}

\hypersetup{
  colorlinks=true,
  linkcolor=osuorange,
  urlcolor=osuorange,
  citecolor=ink
}

\setlength{\parskip}{0.6em}
\setlength{\parindent}{0pt}
\singlespacing

% Headers
\pagestyle{fancy}
\fancyhf{}
\lhead{\footnotesize CareerAI Agent}
\rhead{\footnotesize \thepage}
\renewcommand{\headrulewidth}{0.4pt}

\begin{document}

% ============================================
% TITLE PAGE
% ============================================
\begin{titlepage}
    \centering
    \vspace*{2cm}

    {\Huge\textbf{CareerAI Agent}}

    \vspace{0.5cm}

    {\Large A GenAI-Powered Career Task Automation Tool}

    \vspace{2cm}

    {\large\textbf{Career Preparation Activity}}

    \vspace{1cm}

    {\Large Cameron Brooks}

    \vspace{2cm}

    {\large CS 462 -- Senior Software Engineering Project II}\\[0.3cm]
    {\large Winter 2026}\\[0.3cm]
    {\large Oregon State University}

    \vfill

    \rule{\textwidth}{0.4pt}

    \vspace{0.5cm}

    \textbf{Technology Stack:} SmolLM2-135M $\cdot$ Python $\cdot$ Bash $\cdot$ Gum CLI

\end{titlepage}

% ============================================
% INTRODUCTION
% ============================================
\section{Introduction}

\textbf{CareerAI Agent} is a command-line GenAI tool designed to automate common career preparation tasks. Built as part of the CS 462 Career Preparation Activity, this agent demonstrates how generative AI can assist job seekers with resume analysis, job matching, cover letter generation, interview preparation, and technical assessment feedback.

The agent leverages \textbf{SmolLM2-135M}, a compact 135-million parameter language model from Hugging Face (approximately 180MB), making it lightweight enough to run locally on consumer hardware while still providing useful AI-generated insights.

\subsection{Why SmolLM2?}

SmolLM2 was chosen for several reasons:
\begin{itemize}[noitemsep]
    \item \textbf{Lightweight:} At only 180MB, it runs on CPU without requiring a GPU
    \item \textbf{Fast inference:} Responses generate in under 2 seconds
    \item \textbf{Local execution:} No API keys or internet connection required
    \item \textbf{Open source:} Available on Hugging Face under permissive licensing
\end{itemize}

\subsection{Agent Architecture: Smolagents + MCP}

CareerAI is built using \textbf{smolagents}, Hugging Face's lightweight agent framework that enables LLMs to interact with external tools through the \textbf{Model Context Protocol (MCP)} \cite{smolagents2024}.

The agent connects to the \textbf{JobSpy MCP Server} \cite{jobspymcp2025} to enable real job search across multiple platforms:

\begin{verbatim}
from smolagents import CodeAgent, ToolCollection
from mcp import StdioServerParameters

# Connect to JobSpy MCP server
jobspy_params = StdioServerParameters(
    command="node",
    args=["./jobspy-mcp-server/dist/index.js"]
)

with ToolCollection.from_mcp(jobspy_params) as tools:
    agent = CodeAgent(tools=[*tools], model=model)
    result = agent.run(
        "Find remote software engineer jobs in Portland"
    )
\end{verbatim}

The JobSpy MCP server provides the \texttt{search\_jobs} tool with parameters:
\begin{itemize}[noitemsep]
    \item \texttt{site\_names} -- Platforms to search: Indeed, LinkedIn, Glassdoor, ZipRecruiter, Google Jobs, Bayt, Naukri
    \item \texttt{search\_term} -- Job title or keywords
    \item \texttt{location} -- Geographic location filter
    \item \texttt{hours\_old} -- Filter by posting age (default: 72 hours)
    \item \texttt{results\_wanted} -- Number of results (default: 20)
    \item \texttt{format} -- Output as JSON or CSV
\end{itemize}

This architecture enables the agent to dynamically discover and invoke tools at runtime, following the MCP client-server pattern where the agent acts as the MCP client connecting to external tool servers \cite{mcparchitecture2025}.

% ============================================
% FEATURES
% ============================================
\section{Features}

The CareerAI Agent provides five core career automation features:

\subsection{Resume Analyzer}

Upload a resume file (PDF, DOCX, or TXT) and receive AI-powered feedback including:
\begin{itemize}[noitemsep]
    \item Overall resume score (0--100)
    \item Identified strengths (technical skills, formatting, action verbs)
    \item Areas for improvement (ATS optimization, quantifiable achievements)
    \item Actionable recommendations
\end{itemize}

\subsubsection{Scoring Methodology}

The resume scoring algorithm is based on established Applicant Tracking System (ATS) compatibility research \cite{resume2vec2025, thade2025nlp}. Modern ATS systems use Natural Language Processing (NLP) to extract key information from resumes and score them against job descriptions. The score is computed as a weighted sum across five categories:

\begin{table}[H]
\centering
\begin{tabular}{lcc}
\toprule
\textbf{Category} & \textbf{Weight} & \textbf{Criteria} \\
\midrule
Keyword Density & 25\% & Industry-specific terms, skills matching \\
Action Verbs & 20\% & Strong verbs (``developed,'' ``implemented,'' ``led'') \\
Quantifiable Results & 25\% & Metrics, percentages, numerical achievements \\
Formatting & 15\% & Section headers, consistent structure, length \\
Contact \& Summary & 15\% & Complete info, professional summary present \\
\bottomrule
\end{tabular}
\caption{Resume scoring weight distribution}
\end{table}

The scoring is performed via the \textbf{MagicalAPI Resume Score} API \cite{magicalapi2025}, which provides production-grade resume-to-job matching:

\begin{verbatim}
import requests

def score_resume(resume_path: str, job_description: str) -> dict:
    """Score resume against job description using MagicalAPI."""
    
    api_url = "https://api.magicalapi.com/v1/resume/score"
    headers = {"Authorization": f"Bearer {API_KEY}"}
    
    with open(resume_path, "rb") as f:
        files = {"resume": f}
        data = {"job_description": job_description}
        
        response = requests.post(api_url, headers=headers, 
                                 files=files, data=data)
    
    result = response.json()
    # Returns: {"score": 78, "reason": "Strong skills match...",
    #           "missing_keywords": [...], "strengths": [...]}
    return result
\end{verbatim}

The API uses NLP techniques to analyze skills, experience, certifications, and keyword alignment against the job description. According to Thade et al. \cite{thade2025nlp}, modern ATS scoring algorithms assign weighted points across categories: word count (25 pts max), skills matching (35 pts max), and experience alignment (40 pts max). Resume2Vec \cite{resume2vec2025} demonstrates that transformer-based embeddings can achieve up to 15.85\% improvement over traditional keyword-based ATS systems in candidate matching accuracy.

For local/offline analysis, the agent falls back to the \textbf{APILayer Resume Parser} \cite{apilayer2024}, which extracts structured data (name, email, skills, education, experience) from PDF/DOCX files using NLP, enabling heuristic scoring when the primary API is unavailable.

\begin{figure}[H]
    \centering
    \includegraphics[width=0.85\textwidth]{resume-analyzer.png}
    \caption{Resume Analyzer output showing score, strengths, and improvement areas}
    \label{fig:resume}
\end{figure}

\subsection{Job Matcher}

Enter your skills and preferred location to receive AI-matched job recommendations:
\begin{itemize}[noitemsep]
    \item Skills-based matching algorithm
    \item Match percentage scoring
    \item Salary ranges and company information
    \item Location filtering (remote, hybrid, on-site)
\end{itemize}

\begin{figure}[H]
    \centering
    \includegraphics[width=0.85\textwidth]{job-matcher.png}
    \caption{Job Matcher displaying top 5 matched positions based on skills}
    \label{fig:jobs}
\end{figure}

\subsection{Cover Letter Generator}

Generate personalized cover letters by providing:
\begin{itemize}[noitemsep]
    \item Target company name
    \item Job role/title
    \item The AI generates a professional, customized letter
\end{itemize}

\begin{figure}[H]
    \centering
    \includegraphics[width=0.85\textwidth]{cover-letter.png}
    \caption{Cover Letter Generator creating a personalized application letter}
    \label{fig:cover}
\end{figure}

\subsection{Interview Prep}

Practice with AI-generated interview questions across four categories:
\begin{itemize}[noitemsep]
    \item \textbf{Behavioral:} STAR-method questions about past experiences
    \item \textbf{Technical:} Coding and algorithm questions
    \item \textbf{System Design:} Architecture and scalability questions
    \item \textbf{Culture Fit:} Values and work style questions
\end{itemize}

\begin{figure}[H]
    \centering
    \includegraphics[width=0.85\textwidth]{interview-prep.png}
    \caption{Interview Prep showing behavioral interview questions with tips}
    \label{fig:interview}
\end{figure}

\subsection{Technical Assessment Feedback}

Upload code files to receive AI feedback on:
\begin{itemize}[noitemsep]
    \item Time and space complexity analysis
    \item Code readability score
    \item Strengths and suggestions for improvement
    \item Overall assessment for technical interviews
\end{itemize}

\begin{figure}[H]
    \centering
    \includegraphics[width=0.85\textwidth]{tech-assessment.png}
    \caption{Technical Assessment analyzing code complexity and quality}
    \label{fig:code}
\end{figure}

% ============================================
% ARCHITECTURE
% ============================================
\section{Technical Architecture}

\subsection{Technology Stack}

\begin{table}[H]
\centering
\begin{tabular}{ll}
\toprule
\textbf{Component} & \textbf{Technology} \\
\midrule
Language Model & SmolLM2-135M (Hugging Face) \\
CLI Framework & Gum (Charm.sh) \\
Shell Interface & Bash \\
ML Backend & Python + Transformers + PEFT \\
Fine-tuning & LoRA adapters \\
\bottomrule
\end{tabular}
\caption{CareerAI technology stack}
\end{table}

\subsection{System Flow}

\begin{enumerate}[noitemsep]
    \item User launches \texttt{career\_agent.sh} from terminal
    \item Gum provides interactive menu selection and file picking
    \item User input is processed and formatted as a prompt
    \item SmolLM2 generates contextual AI response
    \item Output is formatted and displayed with rich CLI styling
\end{enumerate}

\subsection{Model Details}

\textbf{SmolLM2-135M} specifications:
\begin{itemize}[noitemsep]
    \item Parameters: 135 million
    \item Model size: $\sim$180MB
    \item Architecture: Transformer decoder
    \item Context length: 2048 tokens
    \item License: Apache 2.0
    \item Source: \url{https://huggingface.co/HuggingFaceTB/SmolLM2-135M}
\end{itemize}

% ============================================
% REFLECTION
% ============================================
\section{Reflection}

\subsection{Usefulness}

The CareerAI Agent demonstrates several practical applications of GenAI for career preparation:

\begin{itemize}
    \item \textbf{Accessibility:} Running locally means no subscription costs or API limits
    \item \textbf{Privacy:} Resume data never leaves the user's machine
    \item \textbf{Speed:} Instant feedback compared to waiting for human review
    \item \textbf{Availability:} 24/7 access without scheduling appointments
\end{itemize}

The tool is particularly useful for:
\begin{itemize}[noitemsep]
    \item First-pass resume review before professional consultation
    \item Interview practice at any time
    \item Exploring job roles that match existing skills
    \item Quick cover letter drafts to customize further
\end{itemize}

\subsection{Limitations}

As with any AI tool, there are important limitations to acknowledge:

\begin{itemize}
    \item \textbf{Generic advice:} Small models produce less nuanced feedback than larger models (GPT-4, Claude)
    \item \textbf{No real job data:} Job matching uses simulated data, not live job postings
    \item \textbf{Context limitations:} 2048 token limit restricts long document analysis
    \item \textbf{No memory:} Each session starts fresh without learning user preferences
    \item \textbf{Hallucination risk:} AI may generate plausible but incorrect suggestions
\end{itemize}

\subsection{Future Improvements}

Potential enhancements for a production version:
\begin{itemize}[noitemsep]
    \item Integration with real job APIs (LinkedIn, Indeed)
    \item Larger model option for more detailed analysis
    \item Resume parsing with structured data extraction
    \item User profiles to personalize recommendations over time
    \item Export functionality for cover letters and analyses
\end{itemize}

% ============================================
% CONCLUSION
% ============================================
\section{Conclusion}

CareerAI Agent successfully demonstrates how lightweight, locally-running GenAI can automate career preparation tasks. Using SmolLM2-135M, the tool provides instant resume feedback, job matching, cover letter generation, interview prep, and code assessment---all from a simple command-line interface.

While not a replacement for professional career counseling or detailed human review, this agent serves as a valuable first-pass tool that can help job seekers iterate quickly and prepare more effectively for their search.

\vspace{1cm}

\begin{center}
\rule{0.5\textwidth}{0.4pt}
\end{center}

\vspace{0.5cm}

\textbf{Repository:} \texttt{aiGn/cli/career\_agent.sh}

\textbf{Run command:} \texttt{./career\_agent.sh}

\textbf{Requirements:} Gum CLI (\texttt{brew install gum}), Python 3.10+, Transformers

% ============================================
% REFERENCES
% ============================================
\begin{thebibliography}{9}

\bibitem{smolagents2024}
Hugging Face. (2024). \textit{smolagents: Lightweight AI Agent Framework}. Retrieved from \url{https://huggingface.co/docs/smolagents/}

\bibitem{jobspymcp2025}
Borgius. (2025). \textit{JobSpy MCP Server: Search Jobs Across Multiple Platforms}. GitHub Repository. Retrieved from \url{https://github.com/borgius/jobspy-mcp-server}

\bibitem{mcparchitecture2025}
Treiber, M. (2025). Model Context Protocol: Inside the MCP Architecture. \textit{LinkedIn Engineering Blog}. Retrieved from \url{https://www.linkedin.com/pulse/model-context-protocol-inside-mcp-architecture}

\bibitem{resume2vec2025}
Sharma, A., et al. (2025). Resume2Vec: Transforming Applicant Tracking Systems with Intelligent Resume Embeddings for Precise Candidate Matching. \textit{Electronics}, 14(4), 794. \url{https://doi.org/10.3390/electronics14040794}

\bibitem{thade2025nlp}
Thade, A., Tate, S., et al. (2025). Resume Analysis Using NLP and ATS Algorithm. \textit{International Journal of Latest Technology in Engineering, Management \& Applied Science (IJLTEMAS)}, XIV(IV), 765--772.

\bibitem{smollm2024}
Hugging Face. (2024). \textit{SmolLM2-135M Model Card}. Retrieved from \url{https://huggingface.co/HuggingFaceTB/SmolLM2-135M}

\bibitem{recruitcrm2024}
Recruit CRM. (2024). \textit{In-depth Guide on Applicant Tracking Systems}. Retrieved from \url{https://recruitcrm.io/blogs/what-is-applicant-tracking-system/}

\bibitem{magicalapi2025}
MagicalAPI. (2025). \textit{Resume Score API: AI-Powered Resume-to-Job Matching}. Retrieved from \url{https://magicalapi.com/resume/score/}

\bibitem{apilayer2024}
APILayer. (2024). \textit{Resume Parser API: NLP-Based Resume Parsing to JSON}. Retrieved from \url{https://apilayer.com/marketplace/resume_parser-api}

\end{thebibliography}

\end{document}
